This section has the aim to explain all acceptance tests done on the mobile app, the reasons why those tests have been chosen and the obtained output.
\subsection{Connection Settings}

\subsubsection{\Large{Test 1}}
To make the app work, the smartphone has been connected to the same wifi of the computer in order to use the local domain. This test has been done because that's the first prerequisite to correctly run the app.\\
\\
\textbf{Given: } domain = local domain found by the command ifconfig -a \\
\textbf{When: } As soon as the user open the app\\
\textbf{Expected then: } the change is saved and  the app go back at the login scene\\
\\
The output is equal to the expected behaviour.

\subsection{Registration}

\subsubsection{\Large{Test 1}}
Check if the registration is stopped if all fields are empty. This test has been done because the user must complete the registration form in order to obtain a personal area.\\
\\
\textbf{Given: } all fileds in the registration form are empty\\
\textbf{When: } the guest wants to register\\
\textbf{Expected then: } the app doesn't have to allow the guest to register\\
\\
The output is equal to the expected then, it appears a popup saying that all fields must be filled in order to register.

\subsubsection{\Large{Test 2}}
Check if the registration is stopped if some fields are empty. This test has been done because the user must fill the whole registration form in order to obtain a personal area.\\
\\
\textbf{Given: } some fields in the registration form are empty, some other have been filled\\
\textbf{When: } the guest wants to register\\
\textbf{Expected then: } the app doesn't have to allow the guest to register\\
\\
The output is equal to the expected then, it appears a popup saying that all fields must be filled in order to register. This test has been done lots of times in order to evaluate if all fields are necessary for the registration or if some of them can be omitted. It has resulted that all fields are necessary in order to obtain a personal area.

\subsubsection{\Large{Test 3}}
Check if the registration is stopped if the SSN is written in a wrong way. This test has been done because the user must provide a possible real SSN; this is important because a third party may required this user data and the way to ask them is by the SSN.\\
\\
\textbf{Given1: } SSN != from format xxx-xx-xxxx \\
\textbf{Given2: } SSN with a correct format but with numbers and letters\\
\textbf{When: } the guest wants to register\\
\textbf{Expected then: } the app doesn't have to allow the guest to register\\
\\
In both cases the output is equal to the expected then, it appears a popup saying that the SSN is not correct.

\subsubsection{\Large{Test 4}}
Check if the registration is stopped if the written SSN is already associated to an existing profile. This test has been done because the user must provide a possible real SSN and it must be unique inside the DB.\\
\\
\textbf{Given: } SSN = one SSN already in DB\\
\textbf{When: } the guest wants to register\\
\textbf{Expected then: } the app doesn't have to allow the guest to register\\
\\
The output is equal to the expected then, it appears a popup saying that the SSN has already been registered.

\subsubsection{\Large{Test 5}}
The user fills all fields with correct values. This is necessary in order to create a new personal area.\\
\\
\textbf{Given: } All fields have been correctly filled\\
\textbf{When: } the guest wants to register\\
\textbf{Expected then: } the app register the user.\\
\\
The output is equal to the expected then, it appears a popup saying that the new user can now login.

\subsubsection{\Large{Test 6}}
The IP hasn't been set in a correct way. This Test has been done to check if the app behave in a proper way when it is not possible to connect to the server (BAD URL)\\
\\
\textbf{Given: } The ip is wrong\\
\textbf{When: } the guest wants to register\\
\textbf{Expected then: } the app doesn't have to allow the guest to register\\
\\
The output is equal to the expected then, it appears a popup saying that there is no active connection between the client and the server.

%LOGIN
\subsection{Login}
\subsubsection{\Large{Test 1}}
Check if the login is stopped if all fields are empty. This test has been done because the user must complete the login form in order to access to his/her personal area.\\
\\
\textbf{Given: } all fileds in the login form are empty\\
\textbf{When: } the guest wants to login\\
\textbf{Expected then: } the app doesn't have to allow the guest to login\\
\\
The output is equal to the expected then, it appears a popup saying that all fields must be filled in order to login.

\subsubsection{\Large{Test 2}}
Check if the login is stopped if some fields are empty. This test has been done because the user must fill the whole login form in order to access to his/her personal area.\\
\\
\textbf{Given: } some fields in the login form are empty, some other have been filled\\
\textbf{When: } the guest wants to login\\
\textbf{Expected then: } the app doesn't have to allow the guest to login\\
\\
The output is equal to the expected then, it appears a popup saying that all fields must be filled in order to login. 

\subsubsection{\Large{Test 3}}
Check if the login is stopped if the SSN is written in a wrong way or is different from any stored in DB. This test has been done because the user has provided a possible real SSN during the registration and the one insert in the login must be equal to any stored in DB.\\
\\
\textbf{Given1: } SSN != from format xxx-xx-xxxx \\
\textbf{Given2: } SSN with a correct format but with numbers and letters\\
\textbf{Given3: } SSN != from any SSN stored in DB\\
\textbf{When: } the guest wants to login\\
\textbf{Expected then: } the app doesn't have to allow the guest to login\\
\\
In all cases the output is equal to the expected then, it appears a popup saying that the SSN format is not correct or that it is not in DB.

\subsubsection{\Large{Test 4}}
Check if the login is stopped if the password is different from the one associated to the given correct SSN. This test has been done in order to verify if the authentication works in a correct way.\\
\\
\textbf{Given: } password != stored password realted to correct SSN\\
\textbf{When: } the guest wants to login\\
\textbf{Expected then: } the app doesn't have to allow the guest to login\\
\\
The output is equal to the expected then, it appears a popup saying that the password is not correct.

\subsubsection{\Large{Test 5}}
The IP hasn't been set in a correct way. This Test has been done to check if the app behave in a proper way when it is not possible to connect to the server (BAD URL)\\
\\
\textbf{Given: } wrong IP\\
\textbf{When: } the guest wants to login\\
\textbf{Expected then: } the app doesn't have to allow the guest to login\\
\\
The output is equal to the expected then, it appears a popup saying that there is no active connection between the client and the server.

\subsubsection{\Large{Test 6}}
The user fills all fields with correct values. This is necessary in order to acces to the personal area.\\
\\
\textbf{Given: } All fields have been correctly filled\\
\textbf{When: } the guest wants to login\\
\textbf{Expected then: } the app login the user\\
\\
The output is equal to the expected then, the scene changes and the user access to his/her personal area.

\subsection{Requests Management}

\subsubsection{\Large{Test 1}}
Check if a third party pending request appears in the Pending request scene. This is a crucial aspect because it is foundamental for the correct working of the web application.  \\
\\
\textbf{Given: } third party request without subscription\\
\textbf{When: } the user is in the pending request area\\
\textbf{Expected then1: } the request appears correctly in the scene and gives the possibility to be accepted or to be denied if a subscribed request hasn't been accepted before \\
\textbf{Expected then2: } the request doesn't appear if a subscribed request has been alreay accepted \\
\\
The output is equal to the expected one when the request is the first one from that specific third party. If the user has already accepted an other request from this third party this request doesn't appear and it's automatically accepted. This behavior is partially different from the expected one in fact it should happen only when a subscribed request has been accepted.\\
Reference to \textit{[R5.1.4]}.

\subsubsection{\Large{Test 2}}
Check if a third party pending subscribed request appears in the Pending request scene. This is a crucial aspect because it is foundamental for the correct working of the web application.\\
\\
\textbf{Given: } third party request with subscription\\
\textbf{When: } the user is in the pending request area\\
\textbf{Expected then1: } the request appears correctly in the scene and gives the possibility to be accepted or to be denied if an other subscribed request hasn't been accepted before \\
\textbf{Expected then2: } the request doesn't appear if a subscribed request has been alreay accepted \\
\\
The output is equal to the expected one when the request is the first one from that specific third party. If the user has already accepted an other request from this third party this request doesn't appear and it's automatically accepted. This behavior is partially different from the expected one in fact it should happen only when a subscribed request has already been accepted.\\
Reference to \textit{[R5.1.4]}.

\subsubsection{\Large{Test 3}}
Check if a third party pending request can be accepted. This is a crucial aspect because it is foundamental for the correct working of the web app.\\
\\
\textbf{Given: }  third party request without subscription in the pending request area\\
\textbf{When: } the user wants to accept the request\\
\textbf{Expected then: } the request disappears from the pending request scene and the Third Party can see the data obtained till that moment \\
\\
The output is equal to the expected one.

\subsubsection{\Large{Test 4}}
Check if a third party pending request can be refused. This is a crucial aspect because it is foundamental for the correct working of the web application.\\
\\
\textbf{Given: }  third party request without subscription in the pending request area\\
\textbf{When: } the user wants to refuse the request\\
\textbf{Expected then: } the request disappears from the pending request scene and the Third Party can see the data obtained till that moment  \\
\\
The output is equal to the expected one.

\subsubsection{\Large{Test 5}}
Check if a third party subscribed pending request can be refused. This is a crucial aspect because it is foundamental for the correct working of the web application.\\
\\
\textbf{Given: }  third party request with subscription in the pending request area\\
\textbf{When: } the user wants to refuse the request\\
\textbf{Expected then: } the request disappears from the pending request scene and the Third Party can't see the data\\
\\
The output is equal to the expected one.

\subsubsection{\Large{Test 6}}
Check if a third party subscribed pending request can be accepted. This is a crucial aspect because it is foundamental for the correct working of the web application.\\
\\
\textbf{Given: }  third party request with subscription in the pending request area\\
\textbf{When: } the user wants to accept the request\\
\textbf{Expected then: } the request disappears from the pending request scene and appears in the subscriber scene and the Third Party can see the data obtained till that moment and new data depending on the chosen frequency \\
\\
The output is equal to the expected one.

\subsubsection{\Large{Test 7}}
Check if a third party subscribed request in the subscriber scene can be removed. This is a crucial aspect because it is foundamental for the correct working of the web application.\\
\\
\textbf{Given: }  third party request with subscription in the subscriber scene\\
\textbf{When: } the user wants to remove the request\\
\textbf{Expected then: } the request disappears from the subscriber scene \\
\\
The output is equal to the expected one.

\subsection{Logout}

\subsubsection{\Large{Test1}}
Check if the logout is done in a correct way. This test is done in order to check if this functionality has been well developed.
\\
\textbf{Given: }  user logged in his/her personal area\\
\textbf{When: } the user wants to logout\\
\textbf{Expected then: } the personal area is closed and it will be necessary to login again in order to access again \\
\\
The output is equal to the expected one.