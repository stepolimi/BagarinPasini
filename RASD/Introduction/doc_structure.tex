This document consist in four sections:
\begin{itemize}
	\item \textbf{Section 1- Introduction:} the aim of this chapter is to give a general introduction and short overview of the 			\textit{TrackMe}'s system by listing goals that must be reached; the scope of the two application and some important semantic 		definitions necessary to fully understand the document.  
	\item \textbf{Section 2 - Overall Description:} the aim of this section is to show a general perspective about the structure and the 		functioning of the software. This is achieved by presenting some descriptive diagrams, the descriptions of the major functions of 		the software and the assumptions under which it will work.
	\item \textbf{Section 3 - Specific Requirements:} the aim of the third section is to specify all requirements necessary to reach 			goals listed in chapter one; provide some possible scenarios which show some possible usages of the two app ; define the use 			cases by providing also some explicative diagrams and describe all possible limitation related to hardware and software. It also 		provide a description of external interfaces, in particular users' and third parties' interfaces are described through some mockups.
	\item \textbf{Section 4 - Alloy:} contains the alloy model and the discussion of its purposes. It also provide a possible world 			generated by the model.
	\item \textbf{Appendix A:} contains the software and tools used, the bibliography and all efforts spent by each group component. 
\end{itemize}