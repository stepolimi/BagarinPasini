In this section is given the alloy model of data requests, which can be both request of data of a single user or request of data of a users' group.\\
It has been choise to model the data requests' use case because, in our opinion, it rappresent one of the most crucial and important activities that can be done throght \textit{trackMe}'s system and because it needs to follow a few constraints which could be open to interpretation if explained by using natural language only.\\
The following contraints are insert in the model:\\
\begin{itemize}
	\item data requests do not exist without a third party;
	\item a group of users doesn't exist if a data of a users' group request hasn't been done by a third party;
	\item each parameter retaled to a request can exist only if there's a request related to it;
	\item a user is in a group of users only if it respects all data requests' constraints.
	\item a group of users' request is accepted if and only if the number of users respecting all request's constraints is equal or grater 		than 1000.
	\item there are no constraints which define when and how a single data request is accepted, it depends from an undefined user 		choise.
\end{itemize}To obtain a possible world the amount of users that must be in a request of a group of users has been drastically decremented.