\subsection{Definitions}
\begin{itemize}
	%def related to the software
	\item{\textbf{TrackMe's system:} the whole software, it's a compact way to refer to all different services that must be developed.}
	
	%def related to data
	\item {\textbf{Data:} location and health parameters collected by the user's device.}
	\item {\textbf{Location:} real time GPS position.}
	\item {\textbf{Health parameters:} data about the user's health status collected by his/her device, such as heart beat, 	vein pressure, body temperature ..}

	%def related to users
	\item {\textbf{User:} an individual user of the application who provides data.}
	\item {\textbf{Third party:} somebody that uses the application because interested in obtaining data of users or groups of users.}
\end{itemize}

\subsection{Acronyms} 
\begin{itemize}
	%def related to database
	\item \textbf{ACID:} Atomicity, Consistency, Isolation and Durability. These are all the logic properties that transactions must 	have to guarantee correct operations on the stored data.
	\item \textbf{DBMS:} the database management system is a software system built to do operations on database in an efficient and correct way.
	\item \textbf{RDBM:} the relational database management system is a particular type of DB based on the relational model introduced by Edgar F. Codd.
\end{itemize}

\subsection{Abbreviation}
\begin{itemize}
	\item {\textbf {RASD:} Requirement Analysis and Specification Document.}
	\item {\textbf {DD:} Design Document.}
	\item {\textbf {D4H:} Data4Help.} 
	\item {\textbf {ASOS:} AutomatedSOS.}
	\item \textbf{DB:} it stands for \textit{database}.
	\item \textbf{App:} it stands for \textit{application}.
\end{itemize}