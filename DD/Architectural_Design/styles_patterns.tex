The following styles and patterns have been used:

\subsubsection{Three layer architecture}
As already said in section 2.1 it has been chosen a three layer architecture formed by the Presentation layer, the Application layer and the Data layer. The use of this architecture allows to phisically divide presentation, application processing and data operations.\\
The presentation layer is extensive and includes both users and third parties interactions in various device (multichannel access) and tools for managing the tier. It is implemented based on the design pattern MVC (Model-View-Controller)  but the direct link to the data layer is connected to the app tier, which is made accessible using services components, consisting of the service model of the application.

\subsubsection{Thin Client for Third Parties}
It has been chosen to use the thin client approach while designing the interaction between third parties' devices and the system. Because of this the main logic is implemented by the App Server, which has been designed to have sufficient computing power and to work in an efficient way. It has also been picked because, by this way, the app via device doesn't take to much space, it can work even in case of limited computing power and it can be updated in an easier way.

\subsubsection{Thick Client for Users}
The thick client approach has been picked while designing the interaction between users' devices and the system. Even though the main logic is still implemented by the App Server, this type of client has been chosen to deal with the associated devices' sensors devoted to collect the user's health parameters. However, the added logic shouldn't require to much space, so also the users' application should still be lightweight and easily updatable.

\subsubsection{Model-View-Controller}
The system is design using the Model-View-Controller (MVC) pattern which separates internal representations of information from the ways information is presented to and accepted from the user. 
\begin{itemize}
	\item The model directly manages data, logic and rules of the application while beeing independent of the user interface.
	\item The controller receives all the users' inputs and performs their requests interacting with the model.
	\item The view gives a representation of the model to the users.
\end{itemize}
The choice of using the MVC pattern is made because goes well with the three layer architecture and garants efficient code reuse and parallel development.

\subsubsection{Singleton}
The model's "service" classes are designed as singletons. Those includes UserService and ThirdPartyService which are the classes in charge of manage the interactions with clients; RequestUserService and RequestGroupService which manages data requestes made by third parties; SubscribeService which gather data from users and group of users and gives them to third parties; HealthParamService and AsosService which are in charge of monitor health parameters of users and send assistance to them if their health parameters goes below a certain threshold. \\
The choice of design these classes as singletons is made because is needed only one object of them to coordinate actions across the system.
