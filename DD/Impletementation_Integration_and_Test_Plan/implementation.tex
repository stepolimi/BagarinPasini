\section{Implementation Plan}
Considering some factors, such as the possible dependencies between different components and modules, it is necessary to provide an order in which would be better to implement the various app's elements. All most important services offered by \textit{D4H} are related to the storage of health parameters which are analyzed in case a user has subscribed to the \textit{ASOS service} to provide tempestive medical assistance; they are also sent to third parties which have required data of group of users or of single users. \\
Starting from the points written above and considering that the implementation plan has also the aim to highlight and modify as early as possible all mistakes and progettual problems found, a track for the order of the components' development is presented in this section.

\begin{itemize}
	\item The model, which corresponds to the application logic component and directly manages data, logic and rules of the 			application.
	\item The persistence unit, which will map each DB's tables and relations to the corresponding objects of the model. 
	\item The controller, which will receive users' inputs and will act accordingly to them on the model.
	\item The view, which will show a representation of the model to the users.
\end{itemize}
\subsection{Model}
	The structure of the model is presented in the 2.2.2 section of this document; note that the UserService and the ThirdPartyService classes in the class diagram in that section won't be part 		of the model but those will be part of the controller. This part of the software will be developed immediatly after the database because it contains the objects that will be mapped to the			DB's tables and relations via JPA and also all the logic and rules of the application, thing that makes this component a very critical one.
\subsection{Persistence unit}
	Via the scope of this unit, it is sensible to develop it after the completition of the model or even in parallel with this one. It is not presented in this document a link map 	beetween the database and the model but it should be simple to deduce it by the model's class diagram in the section 2.2.2 and the database's entity relationship diagram in the section 			2.2.3. The use of OpenJPA gives the possibility to do not implement a real DB because it will be created by the API particular tags.
\subsection{Controller}
	The structure and the interactions of the controller with the model are presented in the section 2.2.2  of this document; as already said, note that the controller's classes in the class 			diagram in that section are the UserService and the ThirdPartyService classes. This part will be implemented after the model because it's scope is to act on it to perform the user's actions, 		so it's the most sensible decision according to the projectual choices made for this software.
\subsection{View}
	The representation of the model that the users will see is presented in the section 3 of this document and also in the section 3 of the RASD. This will be the last part to be developed 			because it represents the least critical and also the most dynamic component of the software.

\section{Testing Plan}
To develop a well working software it is required to do both the structural testing (white box testing) and the functional testing (black box testing) during the implementation and integration of the product. Here below are both analyzed more in the details.

\subsection{Structural testing}
For the coverage of paths, statements, edges and conditions it will be required at least a percentage of 90 for the model's part because of the high risk of errors, faults and failures that lays 	in the nature of this component; for the other parts of the software will be enought a coverage percentage of 80. The process to do the tests must be: 
	\begin{itemize}
		\item Unit testing will be made in parallel with the implementation of each module. It is needed to give particular 					attention to the modules which contains critical algorithms such as the HealthParamService, SubscribeService and 				RequestGroupService classes.
		\item Integration testing will be made after each step of the integration of the modules. In this phase it is needed to 				give particular attention to the correct integration of the modules which are related to the core functions of the system.
		\item System testing will be made after the completion of the implementation and integration steps to test the 					conformity with the functional and non-functional requirements.
		\item Regression testing will be made constantly during the life cycle of the software after it's release. 
	\end{itemize}
\subsection{Functional testing}
	The functional testing is left to the software development team; for the develop of this tests it will be needed again to give particular attention to the limit cases and particular cases 		related to the core functions of the software.
\section{Integration Plan}
The integration order should follow the implementation's one: first of all the model's classes related to the data storage will be integrated with the corresponding database's tables and relations via the persisentence unit, then  the other classes of the model will be integrated with each other and with the UserService and ThirdPartyService and last the view's part and its relations with the server.\\ 
Each component should be integrated with the others only after it's almost totally completion and with a satisfying unit testing's result to avoid and contain the impact of each possible error, fault and failure on the system. For the same scope, right after a component has been integrated with the system, the relative integration testing should be made. 





